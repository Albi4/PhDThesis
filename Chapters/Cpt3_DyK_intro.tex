\chapter{D\lowercase{y}K experiment}
%\section{Overview} depnds how much are the thesis discussed
While ultracold atoms laboratories are in general on the technically heavy side of table-top experiments, the Dy-K machine is a challenging set up even for the standards of the community. While recently the flourishing of ultracold atomic experiments where new, controlled, complexity is added, mostly in the form of \ac{QGM}, use to selectively control individual atoms in lattices or projecting arbitrary potential. The Dy-K set up, while focusing on bulk physics in harmonic traps, proves to be a cumbersome at times.
 For example, a standard ultracold experiment working with an alkali atom in a bulk system, typically uses one or two lasers resonant with the D$2$ and D$1$ lines for cooling, in the \ac{MOT} and grey molasses stages, and one or two high power infrared beam for trapping and evaporative cooling. Working with an heteronuclear mixture requires additional resonant laser sources for the different transitions of the various atomic species. Moreover, lanthanides atoms like dysprosium, erbium, europium \albi{cite} present a richer electronic structure, which can be used to produce colder sub-\unit{\micro\K} atomic samples before evaporative cooling, as is the case in the Dy-K experiment. \albi{this part to be reduced a bit}
 
 When I joined the team at the end of $2021$, the experiment was already fully operational. Many important results were already successfully achieved, in particular regarding the preparation of the doubly degenerate mixture \cite{Ravensbergen2018poa}, the precise measurement of dysprosium polarizability \cite{Ravensbergen2018ado, Kreyer2021mot}, the observation of interspecies Feshbach resonance \cite{Ravensbergen2020rif}, and intr \cite{Soave2022lff}. In this chapter, I will first summarize in Sec.~\ref{Sec:ExpSeq} the procedure developed to obtained the doubly-degenerate Fermi mixture \cite{Ravensbergen2018poa,Ravensbergen2020PhD, Tzanova2020PhD}, while in Sec.~\ref{Sec:DyKFeshbachOld}, I will describe the results already obtained concerning the investigation of interaction in the mixture \cite{Ravensbergen2020rif} and in Dy alone \cite{Soave2022lff}.
 
 \section{Experimental sequence at a glance}
 \label{Sec:ExpSeq}
In our experiment we use on of the two fermionic isotope of dysprosium, $^{161}$Dy, combined with the only fermionic isotope of potassium, $^{40}$K. We employ an all optical approach to cool the mixture to degeneracy. Due to the different magnetic moments of the two atomic species, the magnetic field gradients applied during the \ac{MOT} are very different, requiring a sequential loading. Since the magnetic gradient used for potassium is about $5$ times stronger than the one needed for dysprosium, it is convenient to first load the K \ac{MOT} and then the Dy one, in order to avoid large losses in the Dy cloud. 

Potassium is loaded in to the $3$D \ac{MOT} from its atomic source, a $2$D \ac{MOT} prepared in a separated glass cell. Both $3$D and $2$D {MOT}s operate on the potassium D$2$ line at \SI{767}{nm} (linewidth $\Gamma_{D2}/2\pi\sim\SI{6}{MHz}$), and all the light beams are generated with the same source\footnote{Toptica TA Pro}. After the \ac{MOT}, potassium is further cooled by performing gray molasses on the D$1$ line (wavelength \SI{770}{nm} with linewidth $\Gamma_{D1}/2\pi\sim\SI{6}{MHz}$), reaching a temperature of around \SI{30}{\micro K}, with an almost ($\sim80\%$) spin polarised sample in the lowest hyperfine state $\ket{9/2,-9/2}$. During the grey molasses cooling, the K cloud is transferred in a large volume reservoir crossed \ac{ODT}, operating in the near infrared at a wavelength of \SI{1064}{nm}\footnote{Azurlight ALS-IR-$1064$-$5$-I-CC-SF}. \albi{add some ref on the general preparation of K in other systems, see also pra cornee}

Once potassium is trapped in the reservoir \ac{ODT}, the preparation of the Dy cloud can start. The dysprosium atomic source is a high-temperature effusion oven, operating at around $1000\,^\circ$C. The atomic beam is then decelerated using a Zeeman slower, operating along the broad transition at \SI{421}{nm} ($\Gamma_{421}/2\pi\sim\SI{32}{MHz}$). The slowed atomic beam is capture in a $3$D \ac{MOT} generated at the narrow intercombination transition at \SI{626}nm (linewidth $\Gamma_{626}/2\pi\sim\SI{135}{kHz}$). Thanks to narrowness of transition used in the \ac{MOT}, the cold atomic sample obtained is already spin polarized in the lowest hyperfine state \cite{Dreon2017oca}, that is the stretched state $\ket{21/2, -21/2}$. At the end of the compressed \ac{MOT} stage, the temperature of the atomic cloud is approximately \SI{8}{\micro K}. After the compressed \ac{MOT} the Dy cloud is also transferred in the same reservoir \ac{ODT} of K.

Before the evaporative cooling stage, necessary to reach the degenerate regime, Dy undergoes an additional resonant cooling stage\footnote{This additional stage was implemented in the experiment before I joined the team, but reported for the first time in Ref.~\cite{Ye2022ool}, which is part of this thesis as chapter \ref{Chpt:Lowfieldpaper}.}, performed near the narrow transition at \SI{741}{nm}, with linewidth $\Gamma_{741}/2\pi\sim\,\SI{1.7}{kHz}$ \cite{Lu2011soa}. %\albi{maybe there is something regarding the 626 cooling in Paris?} 
We perform this cooling stage after transferring the atomic clouds in an elongated \ac{ODT}. Thanks to this cooling stage, we manage to reach temperatures around \SI{1}{\micro K} with the Dy cloud. More details on the narrow-line cooling sequence are reported in Sec. \ref{sec:appB}. It is worth noticing here that, while as reported in \cite{Ye2022ool} no effect of the \SI{741}{nm} light are observed on the potassium component, the rapid changes in the trap configuration, strongly affects the potassium cloud, which loses a sizeable amount of atoms. This does not strictly affect the efficiency of evaporative cooling, and has not been a limiting factor in most of the experiments presented in this thesis, but reduces the tunability of the potassium atom number. This limitation and possible solutions are also discussed in Chapter \albi{one of the concluding chapters}.

At the end of the \SI{741}{nm} cooling stage, the atomic mixture is transferred in \SI{300}{ms} in a crossed \ac{ODT} generated at \SI{1064}{nm}. This trap is formed by two tightly focused beams\footnote{Both generated from the Mephisto MOPA $18$ NE}, an horizontal one with waist \SI{25}{\micro m}, and  a vertical one with waist of \SI{60}{\micro m}. The tighter waist of the trap compared  to the reservoir is chosen to increase the density, and improve thermalization. 

Contrary to other fermionic experiments, for highly magnetic lanthanides, it is possible to  perform evaporative cooling with a single spin state thanks to universal dipolar scattering \cite{Lu2012qdd, Aikawa2014rfd}. In our case, we perform evaporative cooling with Dy in its lowest hyperfine state, while also sympathetically cool K. At the end of the evaporative cooling sequence, we are able to reach the deeply degenerate regime, with typical numbers of $N_{\textrm{Dy}}=\SI{1.2e5}{}$, $T_{\textrm{Dy}}/T^{\textrm{Dy}}_F\approx0.13$ and $N_{\textrm{K}}=\SI{2.5e4}{}$, $T_{\textrm{K}}/T^{\textrm{K}}_F\approx0.1$. 

Evaporative cooling in the Dy-K mixture has been found to be most effective for low magnetic field, below \SI{500}{mG}. Above this field, for relative high temperature $\gtrsim\SI{1}{\micro K}$, strong three-body losses in Dy, also observed in Er \cite{Krstajic2025cot}, prevent efficient cooling. A more detailed discussion on Dy intraspecies interaction is reported in the next section, while a discussion regarding the limitations of evaporative cooling in specific is reported in Sec. \albi{second to last chapter}. 


During the years, other trapping beams, at different wavelengths, were aligned on the atoms to explore the influence of the trapping light in our experiments (a good portion of Part \hyperref[pt:MakingProbing]{III} is dedicated to this investigation). For the purpose of the mixture preparation is important to mention here that 

the evaporative cooling sequence, was modified, by adding an additional 






\begin{figure}[thb]
	\includegraphics[trim=0 0 0 2cm, clip, width=1\columnwidth]{gfx/Cpt4_fig/ODT_graph.pdf}
	\caption{\label{fig:ODTscheme}}
\end{figure}

 
 
  \albi{table with the different transition and stages of experimental preparation?}



missing ingredient are the interactions

%Due to the intrinsic explorative nature of our experiment, at times, we had aligned on the mixture up to nine infrared beams used for trapping, combined this with two sets of \ac{MOT} beams and two set of molasses beam

\section{Feshbach resonances in D\lowercase{y}-K}
\label{Sec:DyKFeshbachOld}



